\documentclass{beamer}
\usetheme{metropolis}

\usepackage{polyglossia}
\usepackage{geometry}
\usepackage{mathtools, amssymb, amsfonts}

\usepackage{float}
\usepackage{listings}
\usepackage{graphicx}
\usepackage{xcolor}
\usepackage{tikz}
\usepackage{listings}
\usepackage{tikz}
\usepackage[locale = FR, exponent-product = \cdot, inter-unit-product = .]{siunitx}
\usetikzlibrary{arrows,shapes,calc,angles,decorations.markings,patterns,decorations.pathmorphing}
\usepackage[shortlabels]{enumitem}
\usepackage{comment}

\setdefaultlanguage{french}

\newcommand{\RR}{\mathbb R}
\newcommand{\CC}{\mathbb C}
\newcommand{\QQ}{\mathbb Q}
\newcommand{\ZZ}{\mathbb Z}
\newcommand{\NN}{\mathbb N}
\newcommand{\KK}{\mathbb K}
\newcommand{\FF}{\mathbb F}
\newcommand{\LL}{\mathbb L}
\newcommand{\PP}{\mathbb P}
\newcommand{\EE}{\mathbb E}
\newcommand{\VV}{\mathbb V}

%%%
\theoremstyle{plain}
\newtheorem{exer}{Exercice}


\newtheorem{ques}{Question}

\newtheorem{prop}{Proposition}


\title{{\sffamily Tremplin: Séance 6}}

\date{\today}

\author{W. Jallet}

\begin{document}
	
\begin{frame}
	\maketitle
	\url{https://github.com/ManifoldFR}
\end{frame}

\section*{Combiner suites et fonctions}

\begin{frame}

\textit{Vous connaissez les suites, vous connaissez les intégrales... et les suites d'intégrales ?}

L'\textit{intégrale} d'une fonction continue $f$ sur un segment $[a,b]$ de la droite réelle $\RR$ est le nombre réel noté
\[
\int_a^b f(x)\,dx
\]
Il représente l'aire \textit{algébrique} sous la courbe de la fonction $f$ : comptée \textit{positivement} là où $f$ est positive et \textit{négativement} là où elle est négative.
\end{frame}

\begin{frame}
Une \textit{suite} n'est rien d'autre qu'une succession de nombres réels $(u_n)_{n\in\NN} = (u_0,u_1,\ldots)$, qui peut se définir par une \textit{formule explicite} pour chaque terme ou par une \textit{relation de récurrence}.

\end{frame}

\begin{frame}{Question}
	\textit{<< Est-ce qu'on peut combiner les suites et les intégrales, du coup ? >>}
\end{frame}

\begin{frame}{}
La notion se généralise lorsque l'intervalle d'intégration n'est pas un segment, notamment lorsqu'il est infini ou lorsque la fonction n'est pas défini en son bord.

Soit $a > 0$. Montrer l'existence de la limite (et la calculer) 
\[
\lim_{a\to 0^+} \int_a^1 \frac 1{\sqrt{x}}\,dx
\]\\
Cette limite est appelée \textit{intégrale \textbf{impropre} sur l'intervalle $]0,1]$}, et elle est notée sans surprise
\[
\int_0^1 \frac 1{\sqrt{x}}\,dx
\]
Globalement, les opérations que vous connaissez sur les intégrales restent les mêmes.

\end{frame}

\begin{frame}
En quoi n'est-ce pas la même notion d'intégrale que celle que vous connaissez ?\pause

En effet, la fonction $x\longmapsto \frac{1}{\sqrt x}$ n'est \textbf{pas} continue en 0!

Mais... \textbf{est-ce grave} ?
\end{frame}

\begin{frame}{L'intégrale classique fait des trucs cools}

Considérons la suite d'intégrales définie \textbf{explicitement}\footnotemark par
\[
	I_n = \int_0^1 x^n\,dx
\]\pause

\begin{ques}
Calculer la limite de $I_n$.
\end{ques}

\footnotetext[1]{la plupart du temps, on a affaire à des suites d'intégrales qui est définie explicitement... sauf dans la preuve du théorème de Cauchy-Lipschitz.}

\end{frame}

\begin{frame}

\begin{ques}
	Si on note pour tout $n$ 
	\[
	f_n\colon x\in[0,1] \longmapsto x^n
	\]
	déterminer la limite de $\left(f_n(x)\right)_{n\in\NN}$ pour chaque $x$.
\end{ques}

\end{frame}


\begin{frame}
C'était prévisible...\pause

Il existe même un théorème qui démontre que c'est vrai avec n'importe quelle suite de \textbf{fonctions continues} $f_n$ !\footnotemark 

\footnotetext[2]{\url{https://wikipedia.org/Théorème_de_convergence_dominée}}
\end{frame}

\begin{frame}{Et avec des intégrales \textit{impropres} ?}

On va regarder la suite
\[
J_n = \int_0^{+\infty} e^{-nx}\, dx
\]
d'intégrales des fonctions
\[
	g_n\colon x\in {[0,+\infty[} \longmapsto e^{-nx}.
\]
\end{frame}

\section*{Des suites de nombres complexes}

\begin{frame}
	
\end{frame}





\end{document}